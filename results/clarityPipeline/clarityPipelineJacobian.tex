\documentclass[10pt]{beamer}
\usepackage{verbatim}
\graphicspath{{./images/}}
\newcommand{\jacobianframe}[2]
{
 \begin{frame}

  \begin{columns}
   \begin{column}{0.7\textwidth}
    \includegraphics[width=\textwidth]{#1Jacobian.png} \\
    Average Volume Change: #2
   \end{column}
   \begin{column}{0.3\textwidth}
    \includegraphics[width=\textwidth]{#1JacobianHistogram.png}
   \end{column}
  \end{columns}
 \end{frame}
}

\begin{document}
 \begin{frame}
  \frametitle{Jacobian Determinant}
  \begin{itemize}
   \item Log of \emph{Jacobian Determinant} of LDDMM mapping, $\log |D\phi_{10}|$, measures local volume change
    \begin{itemize}
    \item $>0$ means ARA would need to expand to match CLARITY brain
    \item $<0$ means ARA would need to contract to match CLARITY brain
    \end{itemize}
   \item Following images show the $\log |D\phi_{10}|$ overlaid on coronal, axial and sagittal CLARITY slices
   \item In general most ARA regions contract to match CLARITY brain
   \item Exception is in olfactory bulbs which usually expand during registation
  \end{itemize}
 \end{frame}
 \jacobianframe{Cocaine174}{-13.9\%}
 \jacobianframe{Cocaine175}{2.7\%}
 \jacobianframe{Cocaine178}{-10.3\%}
 \jacobianframe{Control181}{-14.1\%}
 \jacobianframe{Control182}{-25.5\%}
 \jacobianframe{Control239}{-35.1\%}
 \jacobianframe{Control258}{-30.5\%}
 \jacobianframe{Fear197}{-17.6\%}
 \jacobianframe{Fear199}{-43.3\%}
\end{document}